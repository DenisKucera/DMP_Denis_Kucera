\section{Oscar95}
Oscar95 je školní robotické rameno od firmy Elcom Education řízené čtyřmi krokovými 
motory. Rameno bylo zkonstruováno v roce 1995. Původně sloužilo jako pomůcka pro 
výuku automatizace. Dalo se řídit buď ručně pomocí speciálního ovladače s tlačítky a nebo za pomocí programu v počítači. Program v počítači umožňoval studentům si vyzkoušet jednoduché řízení robota pomocí programu. Program mohli studenti psát v jazyce PASCAL, C++, BASIC, nebo také Baltazar. Poslední laboratorní práce s ním byla provedena v roce 2009 -- viz. obrázek \ref{fig:oscar_old}  na straně 
\pageref{fig:oscar_old}. Od té doby čeká na modernizaci. V některých aspektech je vzhledem ke stáří poměrně neaktuální. 
Rameno umožňuje pohyb 360° kolem základny, svírání předních čelistí, pohyb vertikálně a 
horizontálně celou konstrukcí. Rameno je vyrobené převážně z hliníku, až na pár ocelových 
dílů a mosazné závitové tyče, aby byla celá konstrukce, pokud možno co nejlehčí. Dokáže 
uzvednout předmět o hmotnosti až 100 g. Dále disponuje i optickými senzory koncových 
dojezdů, ale o těch více v další podkapitole. Původní deska byla vzhledem ke stáří a 
nekompatibilitě s dnešní pokročilou technikou nahrazena modernější. \cite{Oscar95} \cite{staveb-robot}
\subsection{Krokové motory}
Dalším důležitým dílem je pohon celého ramene. Na Oscarovi jsou osazeny celkem čtyři
krokové motory. Konkrétně se jedná o typ HY100-1713-020 A4 od firmy MAE. Většina 
dnešních moderních strojů například 3D tiskárny, CNC frézky, gravírky nebo lasery v yužívají 
jako pohonnou jednotku právě krokové motory. Největší výhodou a důvodem, proč se používají 
právě v těchto aplikacích, je jejich vysoká přesnost, odolnost a nízká poruchovost. 
K mechanickému tření dochází jenom v ložiscích, proto prakticky nedochází k opotřebení 
motoru. Na rozdíl od běžných motorů se dají ovládat s přesností na jednotlivé stupně jedné 
otáčky. Nepotřebujeme ani enkodéry pro zjištění polohy natočení motoru. Stačí nám pouze 
počítat jednotlivé řídící impulzy. Zároveň se dají pevně zastavit v dané poloze a tím držet svou 
pozici. Pro jejich správné fungování je potřeba použít řídící elektroniku neboli “driver“, který
řídí pohyb motoru. Princip krokového motoru spočívá v postupném spínání cívek statoru a 
speciálně magneticky polarizovaném rotoru. Tento speciální rotor má výstupky po svém 
obvodu. Tyto výstupky mají vždy opačnou polaritu než tělo rotoru. Při sepnutí cívek statoru se 
rotor pootočí právě o vzdálenost jednoho výstupku a tím udělá jeden krok. Při správném spínání 
cívek dochází ke spojitému pohybu v krocích. Jedinou nevýhodou tohoto systému je, že může 
dojít k situaci, kdy se rotor otáčí pomaleji než magnetické pole ve statoru. Tento jev se nazývá 
ztráta kroku a dochází k němu většinou při přetížení motoru, nebo při velmi rychlému spínání 
cívek statoru. Dnešní moderní řídící elektroniky si s tímto problémem dokážou poradit pomocí 
zpětné vazby. \cite{krokove} \cite{bibtex:Kratochvíl}
\subsection{Optické závory}
Jak jsem už dříve zmiňoval, rameno je osazeno optickými senzory dojezdu. Celkově jsou 
použity čtyři. Optické senzory pracují na principu infračervené LED diody a fototranzistoru
nebo fotodiody. Kombinací těchto prvků vznikne součástka zvaná optická závora. Tato 
součástka pracuje na principu infračerveného světelného toku mezi vysílačem (LED diodou) a 
přijímačem (fototranzistorem nebo fotodiodou). Jako zdroj světla je nejčastěji použita
infračervená LED dioda s vlnovou délkou v rozmezí 760 - 940\,nm. Napájecí napětí této 
diody se pohybuje většinou kolem 1,2\,V. Jako přijímač se používá buď fotodioda, nebo častěji 
fototranzistor. Fotodioda se standardně chová jako běžná dioda, pokud ale její přechod 
osvítíme, dojde k nárůstu proudu v závěrném směru. Používanější fototranzistor se naopak 
chová jako běžný bipolární tranzistor jenom s tím rozdílem, že místo báze reguluje proud 
dopadající světlo. Námi nejvíce známé použití optických závor je u pokladen v supermarketech. 
Pokaždé, když se přeruší světelný tok z fotodiody do fototranzistoru motory pohánějící pás se 
zastaví. Dochází tak k pozvolnému posouvání zboží. Infračervené světlo je pro naše oko 
neviditelné, proto není možné tyto děje běžně pozorovat. \cite{bibtex:Kratochvíl}
\subsection{Původní deska}
Oscar95 byl od výroby osazen speciální deskou dělanou na míru pro účely 
automatizovaného ovládání krokových motorů. Deska má tvar pravidelného šestiúhelníku 
s vyříznutou obdélníkovou dírou. Je to oboustranná DPS, ale má osazené pouze vývodové THT 
součástky z jedné strany. Geometrie jednotlivých měděných cest je už také značně zastaralá. 
V dnešní době se už jen málokdy vidí deska s obdélníkovými hranami u měděných cest. Ve své 
době tato deska umožňovala zajímavé funkce. Jednou z nich byla například možnost 
komunikace s počítačem. Ta byla zprostředkována paralelním portem Canon 25. Tento 
konektor zároveň sloužil i pro připojení externího ovladače. Rameno šlo tedy ovládat i ručně. 
Díky dochované dokumentaci k Oscarovi jsem se dozvěděl, že deska byla původně určena pro 
napájení 12\,V. Dále se napájecí napětí stabilizovalo na 5\,V určených pro logiku a optické závory. 
Při rozboru desky jsem zjistil, že se na desce nachází dva integrované obvody L298N a L297, 
které se hojně používají i v dnešní době. Driver L298N má v sobě zabudovaný dvojitý H 
můstek. Obvod L297 zpracovává vstupní signály pro driver L298N. Tyto obvody slouží k řízení 
krokových motorů pomocí TTL logických signálů. Pracovní takt těchto obvodů se nastavoval 
buď externě přes PC, nebo přímo na desce pomocí časovače NE555. \cite{Oscar95} \cite{staveb-robot}  


